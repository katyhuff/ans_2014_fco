% Provide a summary of the work conducted:
%      Describe the technical problem clearly
%      support it with a method

The \Cyclus Fuel Cycle Simulator \cite{carlsen_cyclus_2014} is a framework for 
assessment of nuclear fuel cycle options. While \Cyclus has previously been 
capable of system transitions from the current fuel cycle strategy to a future 
option, those transitions have never previously been driven by market forces in 
the simulation.  This summary describes a set of libraries 
\cite{huff_market_2014} that have been contibuted to the \Cyclus framework to 
enable a market-driven transition analysis.

This simulation framework is incomplete without a suite of dynamically loadable 
libraries representing the process physics of the nuclear fuel cycle (i.e.  
mining, fuel fabrication, chemical processing, transmutation, reprocessing, 
etc.).  Within 
Cycamore \cite{carlsen_cycamore_2014}, the additional modules repository within 
the \Cyclus ecosystem, provides some basic 
libraries to represent these processes. However, extension of \Cyclus with new 
capabilities is community-driven, relying on contributions by user-developers. 
The libraries contributed in this work are examples of such contributions.

\subsection{Motivation}

The attractiveness of a new technology can be assessed to first order by 
evaluating equilibrium fuel cycle scenarios. Equilibrium scenarios are those 
at steady state, in which technologies are deployed statically. However, 
transition dynamics leading up to that equilibrium state also contribute to 
the viability of deploying such a technology \cite{piet_dynamic_2011}.  

% Why did I do the work?
For this reason, fuel cycle scenario analysis is often concerned with the 
transition from one nuclear fuel cycle technology or strategy to another. Such 
analyses are termed ``transition scenarios.''  Transition scenarios seek to 
model the real world dynamics of such a technology shift, in part to inform 
\gls{RDD} decision-making. 

% What will this work show?
Transition scenarios are often modeled from a technology availability 
perspective, deploying new technologies based on their readiness.  The 
capability represented by this work, a transition 
driven by market forces (i.e., material availability) is more realistic. In 
that case, technology readiness may be applied only as a constraint. 

% What were the central motivations and hypotheses?
% The justification for the objectives.
% Why is the work important?

\subsection{Background}
% Who else has done what?
% How?

Previous fuel cycle simulators have achieved market-driven deployment 
capability using look-ahead algorithms for facility deployment 
\cite{schneider_nfcsim:_2005,jacobson_user_2011}.  These simulators typically 
conduct guess-and-check simulation attempts, restarting or crashing if their 
algorithm failed to generate a coherent simulation.

% What has this group previously done?

The development version of the \Cyclus simulator has long been capable of 
technology driven scenarios \cite{gidden_once-through_2012} that relied 
on explicit demand or deployment profiles defined by the user. This capability 
shared the disadvantages of the guess-and-check methods of other simulators, 
since the user-defined deployment profiles may lead to incongruous scenarios 
(i.e., insufficient material or processing availability and unexpected idle 
facilities).  

% Guidance to the reader
By harnessing the \Cyclus dynamic resource exchange paradigm and emphasizing 
generality, the Agents contributed in this work are capable of dynamically 
checking material availability in the simulation and responding accordingly. 
These contributions thereby enable market-driven transition scenarios for a 
range of future use cases in the \Cyclus ecosystem.  

% What should the reader watch for in the paper?
% What are the interesting high points?
% What strategy did we use?

\subsection{Method}
% Summary/Conclusion
% What should the reader expect as conclusion?
% What does it mean?
% What hypotheses do I mean to  test?
% What did I actually test?

Extensions to the \Cyclus framework are necessary because the available
institution, region, and facility archetypes packaged with the Cycamore
repository are not sufficient to model the specific goals of all simulation 
descriptions.  In this case, Facility Agents were contributed to support 
reprocessing and fuel fabrication specifications in the transition scenario
definition of interest and an Institution Agent was contributed to support 
market-driven building and decommissioning of transitioning technologies.  

This is a canonical example of a user-developer's workflow for capability 
extension in \Cyclus. That is, each scenario specification of interest in fuel
cycle analysis is usually sufficiently pathological that modifications must
almost always be made in any simulation framework. This effort demonstrates how 
the modularity built into
the \Cyclus framework allows extension without modification of the core logic.  

