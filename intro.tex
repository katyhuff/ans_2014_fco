% Provide a summary of the work conducted:
%      Describe the technical problem clearly
%      support it with a method


\subsection{Motivation}
% Why did I do the work?
Fuel cycle scenario analysis is often concerned with the transition from one 
reactor technology to another. Such analyses are termed ``transition 
scenarios.''  Transition scenarios seek to model the real world dynamics of 
such a technology transfer. These scenarios are often modeled from a technology 
availability perspective such that new technology is deployed based on its 
readiness.  Arguably, it is more realistic to drive a simulation based on 
market forces such as material availability. In that case, technology readiness 
may be applied only as a constraint. 


% What were the central motivations and hypotheses?
% What will this work show?
% The justification for the objectives.
% Why is the work important?

Fuel cycle transition scenarios inform RD\&D decision-making. While evaluating equilibrium fuel cycle scenarios, in which technologies are sufficiently mature

\subsection{Background}
% Who else has done what?

Fuel cycle simulators have previously acheived this capability using look-ahead
algorithms for facility deployment. 

% How?
These simulators conduct guess-and-check simulation attempts, restarting or
crashing if their algorithm was off.

The Cyclus simulator, due to its dynamic resource exchange paradigm, is capable
of dynamically checking material availability in the simulation and responding
accordingly. 

% What has this group previously done?

The Cyclus simulator has long been capable of technology driven scenarios
\cite{gidden_cyclus_2012}, those scenarios relied on explicit deployment
profiles. This capability is equivalent to the guess-and-check methods of other
simulators, as the user-defined deployment profiles may lead to incongruous
scenarios leading to insufficient material or processing availability and
unexpected idle facilities.  

% Guidance to the reader

This paper will describe the modules added to the Cyclus simulator ecosystem
that enable transition scenarios based on dynamic material availability in the
resource exchange system. By emphasizing generality, this ability has been
contributed to the Cyclus ecosystem as open modules for future use. 

% What should the reader watch for in the paper?
% What are the interesting high points?
% What strategy did we use?

\subsection{Method}
% Summary/Conclusion
% What should the reader expect as conclusion?
% What does it mean?
% What hypotheses do I mean to  test?
% What did I actually test?




Extensions to the Cyclus framework are necessary because the 17 available
institution, region, and facility archetypes packaged with the Cycamore
repository are not quite sufficient to model the specific market-driven
deployment and fuel fabrication specifications in the transition scenario
definition of interest. This is a canonical example of the need for extension
capability in Cyclus. That is, each scenario specification of interest in fuel
cycle analysis is usually sufficiently pathological that modifications must
almost always be made in any simulation framework. The modularity built into
the Cyclus framework allows extension without modification of the core logic.  

In this work, support archetype models have been developed to extend the
capabilities of the Cyclus ecosystem. Those model implementations are described
as are the capabilities that they contribute. 
