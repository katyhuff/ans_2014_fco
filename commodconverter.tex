\subsection{Commodity Converter Facility}

One versatile facility model contributed by this work is a simple
representation of timed commodity transformation. After recieving a resource
such as a material, this facility waits for a user-defined time period. Once
that time period has passed, the resource is offered to the resource exchange
system as a new commodity type. With this behavior, the Commodity Converter
facility is ideal for representing storage.  

In the case of a storage facility, for example, this model requests a commodity 
such as spent fuel, then waits for a cooling period before offering the same 
Material Resource as a cooled spent fuel Commodity.

\begin{table}[h!]
\centering
\begin{tabular}{|l|r|r|r|}
\hline
\textbf{Parameter} & \textbf{Units} & \textbf{Default} & \textbf{Range}\\
\hline
Input Commodity& string & ``'' & any string\\
Output Commodity& string & ``'' & any string\\
Delay Time & months & $0$ & $0-\infty$\\
Storage Capacity & kg & $\infty$ &$0-\infty$ \\
\hline
\end{tabular}
\caption{Input parameters for the Commodity Converter Facility Agent}
\label{tab:commodconverter}
\end{table}


Note that Commodity and Material are distinct concepts in the cyclus framework.
A Material is a subtype of Resource. All Resources have a Commodity type. A
single Material composition in cyclus can therefore be a ``fuel'' or``waste''
Commodity or anything else you want to call it, irrespective of the material
composition. 

