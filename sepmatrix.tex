\subsection{Matrix-Based Separation Facility Model}

By describing the separations process as a simple matrix of efficiencies, a 
simple 1:N material stream transformation is conducted. The specific process 
chemistry for the separation at hand is treated as elemental, as representative 
of a non-laser separations process. The matrix of separation efficiencies has a 
default value: the identity matrix. In this context, the identity matrix 
represents complete and perfect elemental separation without losses. 

\begin{align}
matrix
\label{defaultsep}
\end{align}

Thus, for realistic separations, the user is expected to produce an efficiency 
matrix representing the separations technology of interest to them. 
By requesting the feedstock from the 
appropriate markets, the facility acquires an unseparated feedstock stream. 
Based on the input parameters  in Table \ref{tab:sepmatrix}, the separations 
process proceeds within the timesteps and other constraints of the simulation. 


\begin{table}[h!]
\centering
\begin{tabular}{|l|r|r|r|}
\hline
\textbf{Parameter} & \textbf{Units} & \textbf{Default} & \textbf{Range}\\ 
\hline
& & & \\
\hline
\end{tabular}
\caption{Input parameters for the Matrix-Based Separation Facility Model}
\label{tab:sepmatrix}
\end{table}

Thereafter, separated streams as well as a stream of losses are offered the 
appropriate markets for consumption by other facilities. In the transition 
scenario at hand, the StreamBlender fuel fabridation facility purchases the 
streams it desires in order to produce SFR fuel. 
